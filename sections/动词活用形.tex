\section{动词活用形}

\addcontentsline{toc}{subsection}{动词的种类}
\subsection*{动词的种类}

本讲介绍这些活用形是如何变化的,在实际上是如何构成了各种活用形,举例来说明其特点。每部分首先讲述如何变换,然后讲其实际应用。动词由原形(字典形)变成各种活用形时,\I 类动词(也叫五段动词)、\II 类动词(也叫一段动词)、\III 类动词(即サ变动词、カ变动词)的变化规律是不一样的。因此,看见动词,首先要学会辨别{\bfseries 动词的种类}。

\begin{itemize}
    \item {\bfseries \I 类动词} 指以除る以外的其他う段假名接尾的动词; 以る接尾,る前的假名不在い段或者え段上。
    
    如:買う、喜ぶ、行く、飲む、勝つ、倒す、知る、切る、作る、分かる等等。

    以下特殊词语(以る接尾,る前的假名在い段或者え段上,但为{\I 类动词}):焦る(あせる)、要る(いる)、煎る(いる)、帰る(かえる)、返る(かえる)、限る(かぎる)、切る(切る)、覆る(くつがえる)、蹴る(ける)、遮る(さえぎる)、茂る(しげる)、湿る(しめる)、知る(しる)、滑る(すべる)、散る(ちる)、照る(てる)、握る(にぎる)、練る(ねる)、罵る(ののしる)、入る(はいる)、走る(はしる)、減る(へる)、参る(まいる)、混じる(まじる)、漲る(みなぎる)。

    \item {\bfseries \II 类动词} 指词尾为る,倒数第二个假名为い段或者え段。
    
    例:見る、食べる、考える、助ける、別れる等。

    \item {\bfseries \III 类动词} 又叫サ変动词、カ変动词。サ変动词主要是指する/词干+する(例:洗濯する、掃除する、勉強する)的词汇,カ変动词是指动词来る(くる)。
    
\end{itemize}

\subsection{未然形}

\subsection{连用形}

\subsection{终止形}

\subsection{连体形}

\subsection{假定形}

\subsection{命令形}

\subsection{推量形}



